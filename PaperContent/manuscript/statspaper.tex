\documentclass[12pt]{article}

%% preamble: Keep it clean; only include those you need
\usepackage{amsmath}
\usepackage[margin = 1in]{geometry}
\usepackage{graphicx}
\usepackage{booktabs}
\usepackage{natbib}

% for space filling
\usepackage{lipsum}
% highlighting hyper links
\usepackage[colorlinks=true, citecolor=blue]{hyperref}


%% meta data

\title{Final Title Tbd}
\author{Alexander Rice\\
  Department of Statistics\\
  University of Connecticut
}

\begin{document}
\maketitle

\begin{abstract}
Abstract goes here. Write this soon

\end{abstract}


\section{Introduction}
\label{sec:intro}
Though a relatively new entertainment industry when compared with the film or music industries, video games have proven to bring in a 
significant amount of revenue for their creators and publishers. According to IBISWorld, the size of the video game market was estimated 
as 97.67 billion USD in 2022 ~\citep[(A)][]{Marketsizestat}.  In this blossoming industry, it is important for prospective developers 
to analyze what factors go into making a successful game, considering the sizable budgets and time going into developing games. 

Previous analysis has been done on this topic, namely in Sacranie's paper "Consumer Perceptions \& Video Game Sales: A Meeting of the Minds", 
which focuses mainly on critic reviews in their analysis, while also including variables such as genre and platform in order to construct 
linear and non-linear models for game sales from 2007-2009, concluding that there seems to be a link between game quality and game sales 
~\citep[(B)][]{Sacranie2010Analysis}. In addition, Babb's paper "Comparing Video Game Sales by Gaming Platform" does additional analysis 
utilizing a Kruskal-Wallis test to compare the sales in relation to the hardware they were released on ~\citep[(C)][]{Babb2013Analysis}. 
In this paper, I plan to further the analysis done by others, using a different data set with many more observations over a large period 
of time, and analyzing it using various statistical tests.

% roadmap
The rest of the paper is organized as follows.
The data will be presented in Section~\ref{sec:data}.
The methods are described in Section~\ref{sec:meth}.
The results are reported in Section~\ref{sec:resu}.
A discussion concludes in Section~\ref{sec:disc}.


\section{Data}
\label{sec:data}
The data set being used is called "Video Game Sales Dataset Updated -Extra Feat". It originates from Kaggle, posted by user 
Ibrahim Muhammad Naeem, and contains data compiled from vgcharts containing spanning from 1980 to 2016 ~\citep[(D)][]{maindataset}. 
The data includes information on the names of the games, platform of release, year of  release, genre, publisher, sales numbers 
from different markets, the critic score, the critic count, user score, user count, developer, and rating.

For some elaboration, because the list also contains games that were released on multiple platforms, some names will appear twice. 
Games included in the data set is any game with more than 100,000 units sold. Platform of release is the hardware that the specific 
release could be used with. Year of release is simply the year of initial release, subsequent re-releases will not be specified unless 
it was sold under a different title or on a different platform. Genre is taken from the genre of each game listed on metacritic.com. 
Publisher is the primary publishing company responsible for the games. Sales numbers are given in millions, and has data from North 
America, the EU, Japan, and other regions, as well as all of these pooled into global sales. Critic score, critic count, user score, and user count are 
all pulled from Metacritic. The critic scores are a weighted average of scores from various professional critics and news outlets, with 
critic count simply being the number of reviews taken in by Metacritc. User score and count are similar, simply being the number of 
Metacritic user reviews, along with their average. These statistics taken together are supposed to give some measure of a given games 
quality. Developer is the primary studio responsible for developing the game. Lastly, rating is the recommended age rating given by 
the ESRB, which is required for any major release. There are five different ratings, those being E (everyone), E10+ (everyone ten and up), 
T (teen 13+), M (mature 17+), and AO (adults only 18+).

\section{Methods}
\label{sec:meth}
In terms of design, my plan is to use various types of descriptive statistics to explore the data, and modeling the data using graphs, 
charts, box plots, and other types of visualizations. I plan to use linear regression to attempt to model the effect that platform, 
year of release, genre, publisher, critic and user scores, developer, and rating have on the sales of games throughout different regions 
as well as globally. I will also perform transformations where necessary in order to better fit the data, and properly determine which 
factors, if any, have an impact on sales.


\section{Results}
\label{sec:resu}


\section{Discussion}
\label{sec:disc}

What are the main contributions again?

What are the limitations of this study?

What are worth pursuing further in the future?

\bibliography{refs}
\bibliographystyle{mcap}

\end{document}