\documentclass[12pt]{article}


\usepackage{amsmath}
\usepackage[margin = 1in]{geometry}
\usepackage{graphicx}
\usepackage{booktabs}
\usepackage{natbib}
\usepackage{setspace}
\doublespacing

% highlighting hyper links
\usepackage[colorlinks=true, citecolor=blue]{hyperref}




\title{Predicting Video Game Sales Across Multiple Regions}
\author{Alexander Rice\\
  Department of Statistics\\
  University of Connecticut
}

\begin{document}
\maketitle

\begin{abstract}
Abstract goes here. Write this once the paper is close to or done so then I can give a better picture of what processes are done

\end{abstract}


\section{Introduction}
\label{sec:intro}
Though a relatively new entertainment industry when compared with the film or music industries, video games have proven to bring in a 
significant amount of revenue for their creators and publishers. According to IBISWorld, the size of the video game market was estimated 
as 97.67 billion USD in 2022 ~\citep[(A)][]{Marketsizestat}.  In this blossoming industry, it is important for prospective developers 
to analyze what factors go into making a successful game, considering the sizable budgets and time going into development. 

Previous analysis has been done on this topic, namely in Sacranie's paper "Consumer Perceptions \& Video Game Sales: A Meeting of the Minds", 
which focuses mainly on critic reviews in their analysis, while also including variables such as genre and platform in order to construct 
linear and non-linear models for game sales from 2007-2009, concluding that there seems to be a link between game quality and game sales 
~\citep[(B)][]{Sacranie2010Analysis}. In addition, Babb's paper "Comparing Video Game Sales by Gaming Platform" does additional analysis 
utilizing a Kruskal-Wallis test to compare the sales in relation to the hardware they were released on ~\citep[(C)][]{Babb2013Analysis}. 
In "What Makes a Blockbuster Video Game? An Empirical Analysis of US Sales Data", Cox investigates the effect of multiple different variables,
including year, review, and rating on the total number of sales using OLS regression analysis ~\citep[(E)][]{Cox2013Analysis}.
In this paper, I plan to further the analysis done by others, using a different data set with many more observations over a large period 
of time, and analyzing it using various statistical tests.

% roadmap
The rest of the paper is organized as follows.
The data will be presented in Section~\ref{sec:data}.
The methods are described in Section~\ref{sec:meth}.
The results are reported in Section~\ref{sec:resu}.
A discussion concludes in Section~\ref{sec:disc}.


\section{Data}
\label{sec:data}
The data set being used is called "Video Game Sales Dataset Updated -Extra Feat". It originates from Kaggle, posted by user 
Ibrahim Muhammad Naeem, and contains data compiled from VGChartz containing spanning from 1980 to 2016 ~\citep[(D)][]{maindataset}. 
The data includes information on the names of the games, platform of release, year of  release, genre, publisher, sales numbers 
from different markets, the critic score, the critic count, user score, user count, developer, and rating.

The games included in the data set is any game with more than 100,000 units sold. Platform of release is the hardware that the specific release corresponded to. 
Since the dataset contains games that were released on multiple platforms, some names will appear twice, as platforms are counted as separate entries.
Year of release is simply the year of initial release, subsequent re-releases will not be specified unless 
it was sold under a different title or on a different platform. Genre is taken from the genre of each game listed on metacritic.com. 
Publisher is the primary publishing company responsible for the given entry. Sales numbers are given in millions, and has data from North 
America, the EU, Japan, and all other regions, as well as all of these pooled into global sales. Critic score, critic count, user score, and user count are 
all pulled from Metacritic. The critic scores are a weighted average of scores from various professional critics and news outlets, with 
critic count simply being the number of reviews taken in by Metacritc. User score and count are similar, simply being the number of 
Metacritic user reviews, along with their average. Note that user scores are taken out of ten, while critic scores are taken out of 100. These statistics taken together are supposed to give some measure of a given games 
quality. Developer is the primary studio responsible for developing the game. Lastly, rating is the recommended age rating given by 
the ESRB, which is required for any major release. There are seven different ratings, those being E (everyone), E10+ (everyone ten and up), 
T (teen 13+), M (mature 17+), AO (adults only 18+), EC (early childhood), and K-A (kids to adults). In this dataset however, there
are no AO or EC entries, as these two ratings are very uncommon.

Table 1 contains the summary statistics for the numerical variables. Note that I have omitted entries that had missing data in them, bringing the number of 
entries in the dataset from 16719 observations to 6894 observations. The continuous variables in this dataset are year of release, all of the sales 
numbers, and the ratings and counts from both the critics and users.

\begin{table}[ht]
  \caption{Descriptive Statistics of Continuous Variables}
\label{tab:ds}
\centering
\begin{tabular}{lrrrrrrr}
    \hline
  Variable & Mean & SD & Min & 1st Qu. & Median & 3rd Qu. & Max \\ 
    \hline
    Year of Release & 2007 & 4.236 & 1985 & 2004 & 2007 & 2011 & 2016\\ 
    NA Sales & 0.391 & 0.963 & 0.000 & 0.060 & 0.150 & 0.390 & 41.360\\ 
    EU Sales & 0.235 & 0.684 & 0.000 & 0.020 & 0.060 & 0.210 & 28.960\\ 
    JP Sales & 0.064 & 0.286 & 0.000 & 0.000 & 0.000 & 0.010 & 6.500\\ 
    Other Sales & 0.082 & 0.269 & 0.000 & 0.010 & 0.020 & 0.070 & 10.570\\ 
    Global Sales & 0.772 & 1.955 & 0.010 & 0.110 & 0.290 & 0.750 & 82.530\\ 
    Critic Score & 70.26 & 13.861 & 13.00 & 62.00 & 72.00 & 80.00 & 98.00\\
    Critic Count & 28.84 & 19.195 & 3.00 & 14.00 & 24.00 & 39.00 & 113.00\\
    User Score & 7.184 & 1.439 & 0.500 & 6.500 & 7.500 & 8.200 & 9.600\\ 
    User Count & 174.40 & 584.872 & 4.00 & 11.00 & 27.00 & 89.00 & 10665.0\\
     \hline
  \end{tabular}
  \end{table}

In ____(INSERT WHAT FIGURE THIS ENDS UP BEING HERE) histograms and boxplots of the continous variables for year of release, critic score, and user score.
All three of these seem somewhat normal, but with a skew to the left, much moreso in the case of user and critic scores.
\begin{figure}[tbp]
  \centering
  \includegraphics[width=\textwidth]{histandboxpt1.pdf}
  \caption{Visualizations for year, critic and user score}
  \label{fig:histandboxpt1}
\end{figure}

In ____(INSERT WHAT FIGURE THIS ENDS UP BEING HERE) more histograms and boxplots are included for critic and user count.
Both are skewed heavily to the right, with user count specifically being extremely skewed with many outliers.
\begin{figure}[tbp]
  \centering
  \includegraphics[width=\textwidth]{histandboxpt2.pdf}
  \caption{Visualizations for user and critic count}
  \label{fig:histandboxpt2}
\end{figure}


In ____(INSERT WHAT FIGURE THIS ENDS UP BEING HERE), histograms and boxplots are included for the 5 sales statistics, the response variables.
Note that these are clearly not normal, they are very skewed. 

\begin{figure}[tbp]
  \centering
  \includegraphics[width=\textwidth]{histandboxsales.pdf}
  \caption{Visualizations for each sales statistic}
  \label{fig:histandboxsales}
\end{figure}

To address the lack of normality in figure ___Previous___, figure___thisone___ is included, with the natural log of the sales values.
The boxplots and histograms show much more normal looking data, so this will be important for future analysis. 
Note here that I needed to do a very minor additional transformation to the logarithmic Japan sales numbers. Because so many of the entries
were 0 in the dataset, due to Japan having a smaller population than any of the other regions, I opted to add 1e-10 to each entry before taking
the natural log. This is simply for the sake of modelling, as when there are an overabundance of entries that are all 0, the models will stop working.

\begin{figure}[tbp]
  \centering
  \includegraphics[width=\textwidth]{histandboxlogs.pdf}
  \caption{Visualizations for each logarithmic sales statistic}
  \label{fig:histandboxlogs}
\end{figure}


Figure _______INCLUDE NUMBER ONCE DETERMINED contains visualizations for the categorical variables rating, platform, and genre.
The other two categorical variables, developer and publisher, are very difficult to visualize, having 1305 and 272 levels respectively.
Hence I have opted to not visualize them, as they will not give us a better understanding of what they are. Because of their complexity, 
I will also not be including them in initial statistical analysis, though I will include them in a subsequent test to see if they do 
help with accuracy

\begin{figure}[tbp]
  \centering
  \includegraphics[width=\textwidth]{categoricalplots.pdf}
  \caption{Visualizations for rating using a pie chart, and platform and genre using barplots}
  \label{fig:categoricalplots}
\end{figure}


Table _____INCLUDENUMBER_____ contains a correlation matrix between all numerical variables. All sales statistics are highly
correlated with eachother, which is to be expected. Critic and user score are also highly correlated with each other, indidicating
that usually the critic scores and user scores are similar.

\begin{table}[ht]
  \caption{Correlation Matrix}
  \label{tab:correlation}
  \centering
  \small
  \begin{tabular}{lrrrrrrrrrr}
    \hline
    & GlobalS & CriticS & UserS & Year & CriticC & UserC & NAS & EUS & JPS & OtherS \\
    \hline
    GlobalS & 1.0000 & 0.2371 & 0.0884 & 0.0024 & 0.2907 & 0.2640 & 0.9558 & 0.9392 & 0.6133 & 0.8042 \\
    CriticS & 0.2371 & 1.0000 & 0.5797 & -0.0112 & 0.3949 & 0.2655 & 0.2329 & 0.2128 & 0.1472 & 0.1912 \\
    UserS & 0.0884 & 0.5797 & 1.0000 & -0.2515 & 0.1947 & 0.0185 & 0.0858 & 0.0558 & 0.1279 & 0.0572 \\
    Year & 0.0024 & -0.0112 & -0.2515 & 1.0000 & 0.1970 & 0.1969 & -0.0205 & 0.0380 & -0.0417 & 0.0388 \\
    CriticC & 0.2907 & 0.3949 & 0.1947 & 0.1970 & 1.0000 & 0.3655 & 0.2844 & 0.2660 & 0.1676 & 0.2399 \\
    UserC & 0.2640 & 0.2655 & 0.0185 & 0.1969 & 0.3655 & 1.0000 & 0.2461 & 0.2834 & 0.0725 & 0.2400 \\
    NAS & 0.9558 & 0.2329 & 0.0858 & -0.0205 & 0.2844 & 0.2461 & 1.0000 & 0.8417 & 0.4684 & 0.7269 \\
    EUS & 0.9392 & 0.2128 & 0.0558 & 0.0380 & 0.2660 & 0.2834 & 0.8417 & 1.0000 & 0.5195 & 0.7165 \\
    JPS & 0.6133 & 0.1472 & 0.1279 & -0.0417 & 0.1676 & 0.0725 & 0.4684 & 0.5195 & 1.0000 & 0.3947 \\
    OtherS & 0.8042 & 0.1912 & 0.0572 & 0.0388 & 0.2399 & 0.2400 & 0.7269 & 0.7165 & 0.3947 & 1.0000 \\
    \hline
  \end{tabular}
\end{table}



\section{Methods}
\label{sec:meth}
In terms of design, my plan is to use various types of descriptive statistics to explore the data, and modeling the data using graphs, 
charts, box plots, and other types of visualizations. I plan to use linear regression to attempt to model the effect that platform, 
year of release, genre, publisher, critic and user scores, developer, and rating have on the sales of games throughout different regions 
as well as globally. I will also perform transformations where necessary in order to better fit the data, and properly determine which 
factors, if any, have an impact on sales.

In my first model, I will use the following linear model to model global sales, to act as a baseline linear model that we can adjust to improve accuracy:
$Y_{Global\_Sales}=\hat{\beta_{0}}+\hat{\beta_{1}}*Platform+\hat{\beta_{2}}*Year\_of\_Release+\hat{\beta_{3}}*Genre+\hat{\beta_{4}}*Critic\_Score+\hat{\beta_{5}}*User\_Score+\hat{\beta_{6}}*Rating+\hat{\beta_{7}}*Critic\_Count+\hat{\beta_{8}}*Critic\_Count$
In initial testing, I will be omitting the variables of developer and publisher. They are categorical variables with an extreme amount of variation,
so much so that when tests are ran the presence of them causes the program to slow down significantly, and for potentially important p-values to be left out due to overflow.
****I will be doing one test of them if needed with the global sales, if the model needs more accuracy.****
Next I will explore a transformation of this data. As seen in data, the distribution of the sales statistics are extremely skewed, so I will be taking  
the natural log of the sales, and running all future tests with that. I will be using the following equation:
$Y_{log(Global\_Sales)}=\hat{\beta_{0}}+\hat{\beta_{1}}*Platform+\hat{\beta_{2}}*Year\_of\_Release+\hat{\beta_{3}}*Genre+\hat{\beta_{4}}*Critic\_Score+\hat{\beta_{5}}*User\_Score+\hat{\beta_{6}}*Rating+\hat{\beta_{7}}*Critic\_Count+\hat{\beta_{8}}*Critic\_Count$
Lastly, I am going to reincorporate the publisher and developer variables, and see if that leads to a better fitting model.
$Y_{log(Global\_Sales)}=\hat{\beta_{0}}+\hat{\beta_{1}}*Platform+\hat{\beta_{2}}*Year\_of\_Release+\hat{\beta_{3}}*Genre+\hat{\beta_{4}}*Critic\_Score+\hat{\beta_{5}}*User\_Score+\hat{\beta_{6}}*Rating+\hat{\beta_{7}}*Critic\_Count+\hat{\beta_{8}}*Critic\_Count+\hat{\beta_{9}}*Developer+\hat{\beta_{10}}*Publisher$

NOTE FOR A BIT LATER: Determine at which point to run regression for the other variables: before publisher and developer or after?


\section{Results}
\label{sec:resu}
Preliminary Model(Global):
$Y_{Global\_Sales}=\hat{\beta_{0}}+\hat{\beta_{1}}*Platform+\hat{\beta_{2}}*Year\_of\_Release+\hat{\beta_{3}}*Genre+\hat{\beta_{4}}*Critic\_Score+\hat{\beta_{5}}*User\_Score+\hat{\beta_{6}}*Rating+\hat{\beta_{7}}*Critic\_Count+\hat{\beta_{8}}*Critic\_Count$


\begin{table}[ht]
  \caption{Summary of Preliminary Global Regression Model}
  \label{tab:global}
  \centering
  \small
  \begin{tabular}{lrrrr}
    \hline
    Predictor & Estimate & Std. Error & t value & Pr($>$$|$t$|$) \\ 
    \hline
    (Intercept) & 73.261 & 20.514 & 3.572 & 0.000357 \\
    PlatformDC & -0.844 & 0.511 & -1.652 & 0.098554 \\
    PlatformDS & 0.1037 & 0.1722 & 0.602 & 0.547175 \\
    PlatformGBA & -0.3261 & 0.2079 & -1.568 & 0.116854 \\
    PlatformGC & -0.5071 & 0.1955 & -2.594 & 0.009511 \\
    PlatformPC & -1.079 & 0.1674 & -6.447 & 1.22e-10 \\
    PlatformPS & 0.4491 & 0.2455 & 1.829 & 0.067379 \\
    PlatformPS2 & -0.0835 & 0.1769 & -0.472 & 0.637075 \\
    PlatformPS3 & -0.0911 & 0.1594 & -0.572 & 0.567581 \\
    PlatformPS4 & -0.2880 & 0.1841 & -1.564 & 0.117755 \\
    PlatformPSP & -0.2804 & 0.1772 & -1.582 & 0.113676 \\
    PlatformPSV & -0.3317 & 0.2169 & -1.529 & 0.126233 \\
    PlatformWii & 0.6544 & 0.1690 & 3.873 & 0.000109 \\
    PlatformWiiU & -0.2567 & 0.2350 & -1.092 & 0.274744 \\
    PlatformX360 & -0.2724 & 0.1597 & -1.705 & 0.088192 \\
    PlatformXB & -0.9109 & 0.1939 & -4.697 & 2.69e-06 \\
    PlatformXOne & -0.3337 & 0.2103 & -1.587 & 0.112608 \\
    Year\_of\_Release & -0.03477 & 0.01063 & -3.271 & 0.001078 \\
    GenreAdventure & -0.3312 & 0.1181 & -2.017 & 0.043749 \\
    GenreFighting & -0.1014 & 0.1034 & -0.937 & 0.348901 \\
    GenreMisc & 0.1794 & 0.1035 & 1.657 & 0.097504 \\
    GenrePlatform & 0.0982 & 0.1044 & 0.900 & 0.368295 \\
    GenrePuzzle & -0.4176 & 0.1760 & -2.596 & 0.009448 \\
    GenreRacing & -0.007 & 0.0965 & -0.072 & 0.942370 \\
    GenreRole-Playing & -0.1183 & 0.0849 & -1.394 & 0.163357 \\
    GenreShooter & 0.0708 & 0.0803 & 0.883 & 0.377529 \\
    GenreSimulation & -0.0283 & 0.1199 & -0.236 & 0.813458 \\
    GenreSports & -0.2544 & 0.0908 & -2.800 & 0.005119 \\
    GenreStrategy & -0.3240 & 0.1255 & -2.580 & 0.009887 \\
    Critic\_Score & 0.0455 & 0.0021 & 21.407 & $<$ 2e-16 \\
    User\_Score & -0.1198 & 0.0207 & -5.792 & 7.27e-09 \\
    RatingE & 0.0031 & 0.2263 & 0.014 & 0.988945 \\
    RatingE10+ & -0.2151 & 0.2280 & -0.903 & 0.366814 \\
    RatingK-A & -0.6022 & 1.7850 & -0.357 & 0.720895 \\
    RatingM & -0.3329 & 0.2259 & -1.474 & 0.140613 \\
    RatingT & -0.0899 & 0.2234 & -0.385 & 0.700081 \\
    Critic\_Count & 0.0213 & 0.0015 & 14.477 & $<$ 2e-16 \\
    User\_Count & 0.0007 & 0.0000 & 16.771 & $<$ 2e-16 \\
    \hline
  \end{tabular}
\end{table}

 ***Putting this here temporarily, move this to the discussion later, perhaps put p-value into another smaller table***
At $\alpha=0.05$, the statistically significant predictors are: PlatformGC, PlatformPC, PlatformWii, PlatformXB, Year_of_Release, 
GenreAdventure, GenreMisc, GenrePuzzle, GenreRole-Playing, GenreStrategy, Critic_Score, User_Score, Critic_Count, and User_Count. 
The p-value for the model is $<$ 2.2e-16, meaning that we can conclude that the model is useful. However, the r-squared value for the model is 0.1876,
indicating that only around 19\% of the variability is explained by the model.

Logarithmic Model:
$Y_{log(Global\_Sales)}=\hat{\beta_{0}}+\hat{\beta_{1}}*Platform+\hat{\beta_{2}}*Year\_of\_Release+\hat{\beta_{3}}*Genre+\hat{\beta_{4}}*Critic\_Score+\hat{\beta_{5}}*User\_Score+\hat{\beta_{6}}*Rating+\hat{\beta_{7}}*Critic\_Count+\hat{\beta_{8}}*Critic\_Count$






\section{Discussion}
\label{sec:disc}
NOTE FOR TOMORROW: Change this Discussion and expand on whats there already.
One of the challenges with this data set is simply that it is very large. There are 16,719 entries, and some of them have no meaningful 
information, or are missing information. For example, games that just barely have 100,000 sales have quite limited or no information on 
specific regions, as data was entered as millions only up to two decimal places. Also, there are a sizable number of entries that are 
missing information on scores, developer, and ESRB rating. 
To remedy this, I opted to omit any entries which had NA values, bringing the number of entries down to 6894. 
A consequence of this is that there may have been many relevant entries that were not included in my analysis.
However, due to the sample size being very large even after omitting missing values, I think this data analysis is still relevant.
The entries missing data from my investigation seemed to have no real pattern, I did not notice any common attributes between them.
Also, most of the entries missing data were entries from the end of the entries, which had very low sales and there was an abundance of these.
Thus, I believe we can still regard the results as a reflection of actual sales, even though it doesn't include every entry.
Another limitation of this data set is the information within this data set is not fully up to date, as it does not include entries from 2017 
onward. This is unfortunate, however I believe this analysis will still give relevant information about the past, which can be used as 
indicators in order to predict different aspects of the video game industry moving forward.


\bibliography{refs}
\bibliographystyle{mcap}

\end{document}