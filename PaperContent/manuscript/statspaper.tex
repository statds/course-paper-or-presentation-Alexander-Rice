\documentclass[12pt]{article}

%% preamble: Keep it clean; only include those you need
\usepackage{amsmath}
\usepackage[margin = 1in]{geometry}
\usepackage{graphicx}
\usepackage{booktabs}
\usepackage{natbib}

% for space filling
\usepackage{lipsum}
% highlighting hyper links
\usepackage[colorlinks=true, citecolor=blue]{hyperref}


%% meta data

\title{Final Title Tbd}
\author{Alexander Rice\\
  Department of Statistics\\
  University of Connecticut
}

\begin{document}
\maketitle

\begin{abstract}
Abstract goes here. Write this soon

\end{abstract}


\section{Introduction}
\label{sec:intro}
Though a relatively new entertainment industry when compared with the film or music industries, video games have proven to bring in a 
significant amount of revenue for their creators and publishers. According to IBISWorld, the size of the video game market was estimated 
as 97.67 billion USD in 2022 ~\citep[(A)][]{Marketsizestat}.  In this blossoming industry, it is important for prospective developers 
to analyze what factors go into making a successful game, considering the sizable budgets and time going into developing games. 

Previous analysis has been done on this topic, namely in Sacranie's paper "Consumer Perceptions \& Video Game Sales: A Meeting of the Minds", 
which focuses mainly on critic reviews in their analysis, while also including variables such as genre and platform in order to construct 
linear and non-linear models for game sales from 2007-2009, concluding that there seems to be a link between game quality and game sales 
~\citep[(B)][]{Sacranie2010Analysis}. In addition, Babb's paper "Comparing Video Game Sales by Gaming Platform" does additional analysis 
utilizing a Kruskal-Wallis test to compare the sales in relation to the hardware they were released on ~\citep[(C)][]{Babb2013Analysis}. 
In this paper, I plan to further the analysis done by others, using a different data set with many more observations over a large period 
of time, and analyzing it using various statistical tests.

% roadmap
The rest of the paper is organized as follows.
The data will be presented in Section~\ref{sec:data}.
The methods are described in Section~\ref{sec:meth}.
The results are reported in Section~\ref{sec:resu}.
A discussion concludes in Section~\ref{sec:disc}.


\section{Data}
\label{sec:data}




\section{Methods}
\label{sec:meth}




\section{Results}
\label{sec:resu}


\section{Discussion}
\label{sec:disc}

What are the main contributions again?

What are the limitations of this study?

What are worth pursuing further in the future?

\bibliography{refs}
\bibliographystyle{mcap}

\end{document}