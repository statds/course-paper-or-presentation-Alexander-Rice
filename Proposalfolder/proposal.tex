\documentclass[12pt]{article}

\usepackage{amsmath}
\usepackage[margin = 1in]{geometry}
\usepackage{graphicx}
\usepackage{booktabs}
\usepackage{natbib}

\usepackage{lipsum}
\usepackage[colorlinks=true, citecolor=blue]{hyperref}

%% meta data


\title{Proposal: Analyzing Global Video Game Sales}
\author{Alexander Rice\\
  Department of Statistics\\
  University of Connecticut
}

\begin{document}
\maketitle


\paragraph{Introduction}
Though a relatively new entertainment industry when compared with the film or music industries, video games have proven to bring in a 
significant amount of revenue for their creators and publishers. According to IBISWorld, the size of the video game market was estimated 
as 97.67 billion USD in 2022 ~\citep[(A)][]{Marketsizestat}.  In this blossoming industry, it is important for prospective developers 
to analyze what factors go into making a successful game, considering the sizable budgets and time going into developing games. 

Previous analysis has been done on this topic, namely in Sacranie's paper "Consumer Perceptions \& Video Game Sales: A Meeting of the Minds", 
which focuses mainly on critic reviews in their analysis, while also including variables such as genre and platform in order to construct 
linear and non-linear models for game sales from 2007-2009, concluding that there seems to be a link between game quality and game sales 
~\citep[(B)][]{Sacranie2010Analysis}. In addition, Babb's paper "Comparing Video Game Sales by Gaming Platform" does additional analysis 
utilizing a Kruskal-Wallis test to compare the sales in relation to the hardware they were released on ~\citep[(C)][]{Babb2013Analysis}. 
In this paper, I plan to further the analysis done by others, using a different data set with many more observations over a large period 
of time, and analyzing it using various statistical tests.


\paragraph{Specific Aims}
The main aim to address with the data is to analyze the trends in video game sales. Mainly, I will be attempting to answer the question, 
"What factors go into making a game sell well?" To approach this, I will use a data set containing top selling games, along with 
information pertaining to their release, and analyzing which of these aspects are related to success.

\paragraph{Data}
The data set being used is called "Video Game Sales Dataset Updated -Extra Feat". It originates from Kaggle, posted by user 
Ibrahim Muhammad Naeem, and contains data compiled from vgcharts containing spanning from 1980 to 2016 ~\citep[(D)][]{maindataset}. 
The data includes information on the names of the games, platform of release, year of  release, genre, publisher, sales numbers 
from different markets, the critic score, the critic count, user score, user count, developer, and rating.

For some elaboration, because the list also contains games that were released on multiple platforms, some names will appear twice. 
Games included in the data set is any game with more than 100,000 units sold. Platform of release is the hardware that the specific 
release could be used with. Year of release is simply the year of initial release, subsequent re-releases will not be specified unless 
it was sold under a different title or on a different platform. Genre is taken from the genre of each game listed on metacritic.com. 
Publisher is the primary publishing company responsible for the games. Sales numbers are given in millions, and has data from North 
America, the EU, Japan, and other regions, as well as all of these pooled into global sales. Critic score, critic count, user score, and user count are 
all pulled from Metacritic. The critic scores are a weighted average of scores from various professional critics and news outlets, with 
critic count simply being the number of reviews taken in by Metacritc. User score and count are similar, simply being the number of 
Metacritic user reviews, along with their average. These statistics taken together are supposed to give some measure of a given games 
quality. Developer is the primary studio responsible for developing the game. Lastly, rating is the recommended age rating given by 
the ESRB, which is required for any major release. There are five different ratings, those being E (everyone), E10+ (everyone ten and up), 
T (teen 13+), M (mature 17+), and AO (adults only 18+).


\paragraph{Research Design and Methods}
In terms of design, my plan is to use various types of descriptive statistics to explore the data, and modeling the data using graphs, 
charts, box plots, and other types of visualizations. I plan to use linear regression to attempt to model the effect that platform, 
year of release, genre, publisher, critic and user scores, developer, and rating have on the sales of games throughout different regions 
as well as globally. I will also perform transformations where necessary in order to better fit the data, and properly determine which 
factors, if any, have an impact on sales.



\paragraph{Discussion}
One of the challenges with this data set is simply that it is very large. There are 16,720 entries, and some of them have no meaningful 
information, or are missing information. For example, games that just barely have 100,000 sales have quite limited or no information on 
specific regions, as data was entered as millions only up to two decimal places. Also, there are a sizable number of entries that are 
missing information on scores, developer, and ESRB rating. My plan in order to remedy this is that I will attempt to somewhat limit the 
scope of the data set, by only including games with a certain number of sales within my analysis. I have not decided the cutoff as of 
writing, but I believe it is a necessary step to clean my data analysis. If I can fill any missing entries for rating, score, or 
developer, I will modify the data set in order to have better analysis. Another limitation of this data set is the information within 
this data set is not fully up to date, as it does not include entries from 2017 onward. This is unfortunate, however I believe this 
analysis will still give relevant information about the past, which can be used as indicators in order to predict different aspects 
of the video game industry moving forward. 


\bibliography{refs}
\bibliographystyle{chicago}

\end{document}